% !TeX root = vaje.tex

\chapter{Kotne funkcije}
\label{cha:sin-cos}

\section{Pregled snovi}
\label{sec:sin-cos-pregled-snovi}

Kot smo spoznali, so kotne funkcije nekaj osnovnega za razumevanja tako preproste kot osnovne geometrije. Prav tako so nekaj posebnega tudi grafi kotnih funkcij, katerih periodičnost je iz njih lepo razvidna. \\

Graf, ki predstavlja funkcijo $f(x) = sinx$, imenujemo sinusoida. Pri risanju vedno privzamemo, da je argument $x$ kot, izražen v radianih.

% graf sinusa v osnovni obliki.

Opazimo, da je funkcija sinus periodična funkcija s periodo $2\pi$, ter da je to liha funkcija, torej
\[
sin(-x) = -sin(x)
\]
Sinus je funkcija, ki je definirana na celi realni osi, zavzame pa le vrednosti na intervalu $[-1, 1]$. Kot je razvidno z grafa, so ničle funkcije vsi večkratniki števila $\pi$ in sicer 
\[
x = k\pi, \:   k\in \mathbb{Z}.
\]

Funkcija periodično zavzame tudi maksimume in minimume. V točkah, kjer je 
\[
x = \frac{\pi}{2} + 2k\pi \: ; \: k \in \mathbb{Z},
\]
doseže svoje maksimume ($y = 1$), v točkah pa, kjer je  
\[
x = \frac{3\pi}{2} + 2k\pi \: ; \: k \in \mathbb{Z},
\] 
doseže svoje minimume ($y = -1$).

Očitno je, da je funkcija sinus zvezna.\\

Funkcija $f(x) = cosx$ si je s sinusom zelo podobna.

% graf kosinusa v osnovni obliki
Razlikujeta se pravzaprav le v tem, da je, če opazujemo grafa funkcij, graf kosinusa za $\frac{\pi}{2}$ zamaknjen graf sinusa v levo. Tokrat je funkcija kosinus soda, kar pomeni, da velja
\[
cos(-x) = cosx
\]
Definicijsko območje in zaloga vrednosti sta enaka kot pri sinusu, se pa kosinus razlikuje v ničlah in ekstremih. Funkcija kosinus ima ničle v točkah 
\[
x = \frac{\pi}{2} + k\pi \: ; \: k \in \mathbb{Z},
\]
maksimume zavzame pri
\[
x =2\pi + 2 k\pi, \:   k\in \mathbb{Z}
\]
in minimume pri
\[
x =\pi + 2 k\pi, \:   k\in \mathbb{Z}.
\]





\section{Vaje}
\label{sec:sin-cos-vaje}

%%%%%%%%%%%%%%%%%%%%%%%%%%%%%%%%%%%%%%%%%%%%%%%%%%%%%%%%%%%%%%%%%%%%%%
% Odpremo datoteko, v katero se bodo zapisali odgovori za
% to poglavje.

% Določimo ime datoteke, v katero se bodo pisali odgovori.
% Vsako poglavje mora imeti svojo datoteko.
\def\datotekaOdgovori{odgovori-sincos}

% Odpremo datoteko z odgovori.
\Opensolutionfile{odgovor}[\datotekaOdgovori]

%%%%%%%%%%%%%%%%%%%%%%%%%%%%%%%%%%%%%%%%%%%%%%%%%%%%%%%%%%%%%%%%%%%%%%
% VAJE
%
% Sem vstavimo vaje s pomočjo okolja "vaja". Odgovor napišemo v vajo,
% v okolje "odgovor".

\begin{vaja}
  Narišite graf funkcije, ki je podan s predpisom $f(x) = 3sin(4x)$. Kje ima funkcija ničle, maksimume in minimume, kakšna sta definicijsko območje in zaloga vrednosti?

  \begin{odgovor}
    $0$.
  \end{odgovor}
\end{vaja}

\begin{vaja}
  Podana je funkcija $f(x) = \frac{1}{2} cos(2x)$. Zapišite vsa presečišča grafa funkcije z osjo $x$ in z osjo $y$. Koordinate naj bodo izračunane točno. Zapišite tudi definicijsko območje in zalogo vrednosti. 

  \begin{odgovor}
    Rešitev bi bila tu.
  \end{odgovor}
\end{vaja}

%%%%%%%%%%%%%%%%%%%%%%%%%%%%%%%%%%%%%%%%%%%%%%%%%%%%%%%%%%%%%%%%%%%%%%
% Treba je zapredi datoteko z odgovori

\Closesolutionfile{odgovor}

%%%%%%%%%%%%%%%%%%%%%%%%%%%%%%%%%%%%%%%%%%%%%%%%%%%%%%%%%%%%%%%%%%%%%%
% Odgovori

\section{Odgovori}
\label{sec:sincos-odgovori}

% Vključimo odgovore.
\input{\datotekaOdgovori}


%%% Local Variables:
%%% mode: latex
%%% TeX-master: "vaje"
%%% End:
